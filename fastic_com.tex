\chapter{FastIC+ packet structure}
%
\subsection{Data packet}
%
Each detection event on each channel of the FastIC+ is represented by a 48 bit data packet containing the digitized \emph{ToA} and \emph{ToT} values. The stream of packets is then encoded into Aurora data blocks. A structure of the packet is depicted in Figure \ref{fig:packet}.
\\
\FloatBarrier
\begin{figure}[tph!]
    \begin{center}
        \begin{bytefield}[endianness=little,bitwidth=4em]{8}
            \bitheader{0,1,2,3,4,5,6,7} \\
            \bitbox{4}{\texttt{\footnotesize CHANNEL}} & \bitbox{2}{\texttt{\footnotesize TYPE}} & \bitbox[ltr]{2}{} \\
            \bitbox[lr]{8}{\texttt{\footnotesize TIMESTAMP}} \\
            \bitbox[lr]{8}{\texttt{\footnotesize (22 bits)}} \\
            \bitbox[lrb]{4}{} & \bitbox[lrt]{4}{} \\
            \bitbox[lr]{8}{\texttt{\footnotesize PULSE WIDTH (14 bits)}}  \\
            \bitbox[lrb]{2}{} & \bitbox{1}{\texttt{\footnotesize DBG}} & \bitbox{1}{\texttt{\footnotesize CHP}} & \bitbox{1}{\texttt{\footnotesize TYP}} &
            \bitbox{1}{\texttt{\footnotesize TSP}} & \bitbox{1}{\texttt{\footnotesize PWP}} & \bitbox{1}{\texttt{\footnotesize PAR}} 
        \end{bytefield}
    \end{center}
    \caption{FastIC+ data packet structure}
    \label{fig:packet}
\end{figure}
%
\noindent The \verb|CHANNEL| field specifies the number of the channel that the event was detected on (0-7) or the trigger channel (8). The \verb|TYPE| field indicates the packet type where the following types are recognized:
\begin{itemize}
    \item \emph{ToA + non-linear ToT} type represented by the value \verb|00|,
    \item \emph{ToA only} type represented by the value \verb|01|,
    \item \emph{Linear ToT only} type represented by the value \verb|10|,
    \item \emph{ToA + linear ToT} type represented by the value \verb|11|.
\end{itemize}
%
\verb|TIMESTAMP| and \verb|PULSE WIDTH| fields contain the digitized ToA and ToT values. \verb|DBG| is a flag used in debug mode. The rest of the bits are even parity bits, namely:
\begin{itemize}
    \item \verb|CHP| parity bit of the \verb|CHANNEL| field,
    \item \verb|TYP| parity bit of the \verb|TYPE| field,
    \item \verb|TSP| parity bit of the \verb|TIMESTAMP| field,
    \item \verb|PWP| parity bit of the \verb|PULSE WIDTH| field,
    \item and the overall parity bit \verb|PAR| computed from all fields.
\end{itemize}

\subsection{Statistics packet}
In addition to the data packets, FastIC+ also periodically transmits a statistics packet that contains basic information about the events. Structure of the packet is shown in Figure \ref{fig:statpacket}. This packet is sent onto the Aurora bus in two \emph{K-Blocks} due to the packet size being bigger than the \emph{K-Block} capacity. The first 56 bits are sent in a block with the \emph{BTF} of \verb|0x99|, the rest is sent in a second block with the \emph{BTF} of \verb|0x55|.
\\
\FloatBarrier
\begin{figure}[tph!]
    \begin{center}
        \begin{bytefield}[endianness=little,bitwidth=4em]{8}
            \bitheader{0-7} \\
            \bitbox[lrt]{8}{\texttt{\footnotesize FIFO DROP}}\\
            \bitbox[lr]{8}{\texttt{\footnotesize (20 bits)}}\\
            \bitbox[lrb]{4}{} & \bitbox[lrt]{4}{}\\
            \bitbox[lr]{8}{\texttt{\footnotesize PWIDTH DROP}}\\
            \bitbox[lrb]{8}{\texttt{\footnotesize (20 bits)}}\\
            \bitbox[lrt]{8}{\texttt{\footnotesize DCOUNT DROP}}\\
            \bitbox[lr]{8}{\texttt{(\footnotesize 20 bits)}}\\
            \bitbox[lrb]{4}{} & \bitbox[lrt]{4}{}\\
            \bitbox[lr]{8}{\texttt{\footnotesize TRIGGER DROP}}\\
            \bitbox[lrb]{8}{\texttt{\footnotesize (20 bits)}}\\
            \bitbox[lrt]{8}{\texttt{\footnotesize PULSE ERROR}}\\
            \bitbox[lrb]{8}{\texttt{\footnotesize (16 bits)}}
        \end{bytefield}
    \end{center}
    \caption{FastIC+ statistics packet structure}
    \label{fig:statpacket}
\end{figure}
%
\noindent The packet consists of the following fields:
\begin{itemize}
\item \verb|FIFO DROP| field containing a number of events dropped in the FIFO buffer.
\item \verb|PWIDTH DROP| field containing a number of events dropped due to the pulse width being outside of the specified range.
\item \verb|DCOUNT DROP| field holding the value of dark counts. This field is valid only if \emph{High-Energy Resolution} mode and \emph{Bandwidth optimization} is enabled. 
\item \verb|TRIGGER DROP| field holding the number of packets dropped due to external trigger pulse not being validated. 
\item \verb|PULSE ERROR| field being a malformed pulse counter. Malformed pulse is detected when two consecutive edges are read by the TDC. This may occur due to the limitation of one leading+falling edge per clock period.
\end{itemize}
\newpage
%
\subsection{Counter Extension + Event Counter packet}
When no detection events are processed during the \emph{Coarse Counter Extension} inactivity period, the timestamp counter would overflow the timestamp field. The following packet, specified in Figure \ref{fig:extpacket}, is transmitted periodically so the receiver can base the data timestamp on the most recent coarse counter value, thus drastically increase the dynamic range of the time detection. The packet is encoded in an \emph{K-Block} with the \emph{BTF} of \verb|0xD2|.
\\
\FloatBarrier
\begin{figure}[tph!]
    \begin{center}
        \begin{bytefield}[endianness=little,bitwidth=4em]{8}
            \bitheader{0-7} \\
            \bitbox[lrt]{8}{\texttt{\footnotesize PACKET COUNT}}\\
            \bitbox[lr]{8}{\texttt{\footnotesize (23 bits)}}\\
            \bitbox[lrb]{7}{} & \bitbox[lrt]{1}{}\\
            \bitbox[lr]{8}{\texttt{\footnotesize COARSE COUNTER}}\\
            \bitbox[lr]{8}{\texttt{\footnotesize (24 bits)}}\\
            \bitbox[lrb]{7}{} & \bitbox{1}{\texttt{\footnotesize RST}}
        \end{bytefield}
    \end{center}
    \caption{FastIC+ counter extension and event counter packet structure}
    \label{fig:extpacket}
\end{figure}
%
\noindent The packets fields are:
\begin{itemize}
    \item \verb|PACKET COUNT| field, which holds the number of data packets transmitted since the last reset event.
    \item \verb|COARSE COUNTER| field holding the timestamp from the coarse counter.
    \item \verb|RST| field indicating, that the coarse counter has been reset during the coarse counter packet period.  
\end{itemize}
%