\chapter{FastIC+}
\label{sec:fastic}
The FastIC+ is a configurable ASIC for fast timing applications, featuring an 8-channel front-end for photodetectors such as \emph{SiPM}, \emph{PMT}, or \emph{MCP}, capable of precisely measuring the Time-of-Arrival (\emph{ToA}) and Time-over-Threshold (\emph{ToT}) of photons hitting the detectors. This feature set finds its use in applications that require precise photon timestamping, such as Time-of-Flight Positron Emission Tomography, high-energy physics, mass spectrometry, or LIDAR applications. The ASIC is developed in the \SI{65}{\nano\meter} technology by the Institut de Ciències del Cosmos of the University of Barcelona in close collaboration with CERN. \cite{ficDatasheet}

The ASIC can mostly be used without any additional circuitry. Only power and a \SI{40}{\mega\hertz} low-jitter reference clock need to be provided for correct operation. The chip is configured over an I2C interface. \cite{ficDatasheet}

\FloatBarrier
\begin{figure}[htp!]
    \centering
    \includegraphics[height=4.8cm]{05_FASTICPLUS_BLOCK.pdf}
    \caption{Top-level architecture block diagram}
    \label{fig:fastic_top_level}
\end{figure}
\FloatBarrier



\section{Detection}
Each channel of the ASIC consists of a low-impedance input stage, working with both positive and negative polarity sensors with a dynamic range of \SI{5}{\micro\ampere} to \SI{20}{\milli\ampere} and stability across sensor capacitances from \SI{10}{\pico\farad} to \SI{1}{\nano\farad}. This stage generates three differently scaled replicas of the input current pulse and forwards them to the three signal paths: time, energy, and trigger. \cite{ficDatasheet}
%
\subsection{Time path}
The time path acts as a simple discriminator that compares the pulse to a programmable threshold, which can be set down to a single photoelectron level. The leading edge of the comparator output pulse provides the \emph{ToA} timestamp. The logical OR of all time channels can be output via the \verb|TIME| output. \cite{ficDatasheet}
%
\subsection{Energy path}
The energy path extracts the pulse peak to estimate the energy deposited in the sensor. The chain consists of a transimpedance amplifier, shaper, peak detector with hold, and a comparator that compares the peak detector level with a linear ramp. The length of the output pulse then directly encodes the energy of the pulse. The input can also be compared to a constant threshold to provide a non-linear \emph{ToT} instead. \cite{ficDatasheet}

\subsection{Trigger path}
The final path, the trigger, generates either a low-level trigger per channel or a cluster trigger. The low-level trigger is a logical OR of all the trigger comparators for all channels, whereas the cluster trigger results from an analog summation of the input pulses passed through a single comparator. This trigger can be used internally to trigger the conversion \emph{FSM} or output on a pin. An external trigger can also be injected through a pin. \cite{ficDatasheet}

%Three different measurement modes, 
%\begin{itemize}
%    \item ToA (non-linear ToT),
%    \item Energy (from ToT),
%    \item Hybrid (ToA + Energy),
%\end{itemize}
%are present to allow the user to optimize for measurement of ToA, ToT or both. The maximum detection rate of the impacts is approximately \SI{2}{\mega\hertz} in the linear mode and \SI{50}{\mega\hertz} in non-linear mode.


\section{Digitizing}
\label{sec:fastic_digitizing}
The time, energy, and trigger pulses from each channel are passed to a Front-End Readout block (\emph{FERO}), which interpolates the input pulse and generates a digital representation of the rising edge (with a \SI{25}{\pico\second} time bin) and the falling edge (with a \SI{390}{\pico\second} time bin). \cite{ficDatasheet}

After capturing the edges, the timing information is sent to the Back-End Readout Block (\emph{BERO}), which processes the data. It corrects signal errors, validates and filters triggers, and encodes the timing information into a suitable binary format. Finally, it stores the encoded packet in the channel's \emph{FIFO} buffer. An arbiter with a multiplexer (\emph{MUX}) then manages requests from all the channel \emph{FIFOs} and stores the packets in a global \emph{FIFO}. \cite{ficDatasheet}

In the final step, an \emph{Aurora} serializer converts the packets from the \emph{FIFO} into an \emph{Aurora 64B/66B} serial stream and transmits them over the high-speed \emph{SLVS} output lines. Statistical and counter information is also transmitted if enabled. The stream can be configured to operate at speeds ranging from \SI{80}{\mega\bit\per\second} to \SI{1.28}{\giga\bit\per\second}, in compliance with the \emph{Aurora} specification. \cite{ficDatasheet}

%\section{Configuration and testing}
%An I2C interface has been implemented alongside the Aurora bus to allow for easy configuration of the chip parameters. This interface can run at speeds up to \SI{1}{\mega\hertz} and can be shared between multiple FastIC+ chips. 

%For testing purposes, four debug pins are exposed to read out the state of the internal FSM. If not used for debugging, they can be utilized for configuration of the I2C address. 
