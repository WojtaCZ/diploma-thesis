
\chapter{Aurora protocol}
Aurora 64B/66B is a link-layer, point-to-point protocol that enables high-speed data transfers between two Aurora partners. The Aurora channel, established between the partners, can comprise one or more simplex or full-duplex lanes. Data frames are transmitted in 66-bit blocks, consisting of a two-bit synchronization preamble and 64 bits of data. The channel is shared for both data and control messages. 

Frame notation and bit order adopted from the Aurora specification \cite{auroraSpec} is used in this chapter, the leftmost bit being the first one transmitted to the bus is considered MSB. The rightmost one LSB.
%
\section{Encoding}
As mentioned above, the Aurora protocol utilizes the commonly used 64B/66B encoding. This encoding scheme converts 64 bits of data into a 66-bit block by appending a two-bit preamble. The \verb|0b01| preamble signifies that the block contains 8 octets of data and is therefore called a \emph{Data block}. If the preamble is \verb|0b10|, the block is recognized as \emph{Control block} with the most significant octet in the block being a \emph{BTF} (Block Type Field). The remaining 7 octets are data. Preambles \verb|0b00| and \verb|0b11| are interpreted as errors. These preambles are than used by the receiver to synchronize to the data stream.
\newline\newline
FastIC+ is capable of transmitting the \emph{Data block} and four of the \emph{Control block} types:
\begin{itemize}
    \item \emph{Separator-7} block indicated by the \emph{BTF} value \verb|0xE1|,
    \item \emph{Separator} block indicated by the \emph{BTF} value \verb|0x1E|,
    \item \emph{User K-Block} with \emph{BTF} value indicating the ID of the block (see \ref{sec:kblock}),
    \item \emph{Idle} block indicated by the \emph{BTF} value \verb|0x78|.
\end{itemize}

\subsection{Data block}
A \emph{Data block}, shown in Figure \ref{fig:data}, consists of eight octets of raw data. Data blocks are transmitted only when eight or more octets of data are available. For smaller packets, the \emph{Separator-7} and \emph{Separator} blocks are used.
\\
\FloatBarrier
\begin{figure}[!htpb]
    \begin{center}
        \begin{bytefield}[endianness=little,bitwidth=0.8em, bitheight=1.2em]{32}
            \bitheader{0,1,2,9,10,17,18,25,26,33} \\
            \bitbox{1}{\texttt{\footnotesize 0}} & \bitbox{1}{\texttt{\footnotesize 1}} & \bitbox{8}{\texttt{\footnotesize D7}} &
            \bitbox{8}{\texttt{\footnotesize D6}} & \bitbox{8}{\texttt{\footnotesize D5}} & \bitbox{8}{\texttt{\footnotesize D4}}\\[3ex]
            \hfill
            \bitheader[lsb=34]{34,41,42,49,50,57,58,65} \\
            \hfill
            \bitbox{8}{\footnotesize D3} & \bitbox{8}{\texttt{\footnotesize D2}} & \bitbox{8}{\texttt{\footnotesize D1}} & \bitbox{8}{\texttt{\footnotesize D0}}
        \end{bytefield}
        \caption{Data block structure}
        \label{fig:data}
    \end{center}
\end{figure}

\subsection{Separator-7 block}
A \emph{Separator-7 block} is transmitted when only seven octets of data remain to be sent over the interface. The block consists of the \emph{BTF} indicating the  \emph{Separator-7} type and seven valid octets of data as shown in Figure \ref{fig:sep7}.
\\
\FloatBarrier
\begin{figure}[!htpb]
    \begin{center}
        \begin{bytefield}[endianness=little,bitwidth=0.8em, bitheight=1.2em]{32}
            \bitheader{0,1,2,9,10,17,18,25,26,33} \\
            \bitbox{1}{\texttt{\footnotesize 1}} & \bitbox{1}{\texttt{\footnotesize 0}} & \bitbox{8}{\texttt{\footnotesize 0xE1}} &
            \bitbox{8}{\texttt{\footnotesize D6}} & \bitbox{8}{\texttt{\footnotesize D5}} & \bitbox{8}{\texttt{\footnotesize D4}}\\[3ex]
            \hfill
            \bitheader[lsb=34]{34,41,42,49,50,57,58,65} \\
            \hfill
            \bitbox{8}{\texttt{\footnotesize D3}} & \bitbox{8}{\texttt{\footnotesize D2}} & \bitbox{8}{\texttt{\footnotesize D1}} & \bitbox{8}{\texttt{\footnotesize D0}}
        \end{bytefield}
        \caption{Separator-7 block structure}
        \label{fig:sep7}
    \end{center}
\end{figure}
\newpage
\subsection{Separator block}
A \emph{Separator block} is transmitted when 0-6 octets of data remain to be sent. The block, shown in Figure \ref{fig:sep}, consists of the \emph{BTF} indicating the \emph{Separator} type and a field indicating the number of valid data octets. The order of the valid octets is from LSB to MSB. 
\\
\FloatBarrier
\begin{figure}[!htpb]
    \begin{center}
        \begin{bytefield}[endianness=little,bitwidth=0.8em, bitheight=1.2em]{32}
            \bitheader{0,1,2,9,10,17,18,25,26,33} \\
            \bitbox{1}{\texttt{\footnotesize 1}} & \bitbox{1}{\texttt{\footnotesize 0}} & \bitbox{8}{\texttt{\footnotesize 0x1E}} &
            \bitbox{8}{\texttt{\footnotesize VALID OCTETS}} & \bitbox{8}{\texttt{\footnotesize D5}} & \bitbox{8}{\texttt{\footnotesize D4}}\\[3ex]
            \hfill
            \bitheader[lsb=34]{34,41,42,49,50,57,58,65} \\
            \hfill
            \bitbox{8}{\texttt{\footnotesize D3}} & \bitbox{8}{\texttt{\footnotesize D2}} & \bitbox{8}{\texttt{\footnotesize D1}} & \bitbox{8}{\texttt{\footnotesize D0}}
        \end{bytefield}
        \caption{Separator block structure}
        \label{fig:sep}
    \end{center}
\end{figure}

\subsection{User K-block}
\label{sec:kblock}previouslyto 8, corresponding to \emph{BTFs} \verb|0xD2|, \verb|0x99|, \verb|0x55|, \verb|0xB4|, \verb|0xCC|, \verb|0x66|, \verb|0x33|, \verb|0x4B| and \verb|0x87|, respectively.
\\
\FloatBarrier
\begin{figure}[!htpb]
    \begin{center}
        \begin{bytefield}[endianness=little,bitwidth=0.8em, bitheight=1.2em]{32}
            \bitheader{0,1,2,9,10,17,18,25,26,33} \\
            \bitbox{1}{\texttt{\footnotesize 1}} & \bitbox{1}{\texttt{\footnotesize 0}} & \bitbox{8}{\texttt{\footnotesize BTF}} &
            \bitbox{8}{\texttt{\footnotesize D6}} & \bitbox{8}{\texttt{\footnotesize D5}} & \bitbox{8}{\texttt{\footnotesize D4}}\\[3ex]
            \hfill
            \bitheader[lsb=34]{34,41,42,49,50,57,58,65} \\
            \hfill
            \bitbox{8}{\texttt{\footnotesize D3}} & \bitbox{8}{\texttt{\footnotesize D2}} & \bitbox{8}{\texttt{\footnotesize D1}} & \bitbox{8}{\texttt{\footnotesize D0}}
        \end{bytefield}
        \caption{User K-Block structure}
        \label{fig:kblock}
    \end{center}
\end{figure}
\newpage
\subsection{Idle block}
Figure \ref{fig:idle} shows a structure of an \emph{Idle block}. When no data is available for transmission, \emph{Idle} blocks are transmitted instead to maintain the receivers ability to remain in sync and recover the data clock.
\\
\FloatBarrier
\begin{figure}[h!]
    \begin{center}
        \begin{bytefield}[endianness=little,bitwidth=0.8em, bitheight=1.2em]{32}
            \bitheader{0,1,2,9,10,11,12,13,14,15,16,17,18} \\
            \bitbox{1}{\texttt{\footnotesize 1}} & \bitbox{1}{\texttt{\footnotesize 0}} & \bitbox{8}{\texttt{\footnotesize 0x78}} &
            \bitbox{1}{\texttt{\tiny CC}} & \bitbox{1}{\texttt{\tiny CB}} & \bitbox{1}{\texttt{\tiny NR}} & \bitbox{1}{\texttt{\tiny SA}} & \bitbox{1}{\texttt{\footnotesize 0}} & \bitbox{1}{\texttt{\footnotesize 0}} & \bitbox{1}{\texttt{\footnotesize 0}} & \bitbox{1}{\texttt{\footnotesize 0}} & \bitbox[ltb]{16}{\texttt{\footnotesize Don't care}}\\[3ex]
            \hfill
            \bitheader[lsb=34]{65} \\
            \hfill
            \bitbox[tbr]{32}{\texttt{\footnotesize Don't care}}
        \end{bytefield}
    \end{center}
    \caption{Idle block structure}
    \label{fig:idle}
\end{figure}
\FloatBarrier
%
%
\noindent
The \emph{Idle block} contains four flag bits determining the type of the block:
\begin{itemize}
    \item \verb|CC| bit indicating that the block is \emph{Clock Compensation} idle,
    \item \verb|CB| bit indicating that the block is \emph{Channel Bonding} idle,
    \item \verb|NR| bit indicating that the block is a \emph{Not Ready} idle,
    \item \verb|SA| bit indicating that the receivers obey the \emph{Strict Alignment} rules.
\end{itemize}

\section{Scrambling}
Whenever either \emph{Data Block} or \emph{Control Block} is transmitted, the eight octets in the block have to be scrambled with the polynomial shown in Equation \ref{eq:scrambler}. The scrambling serves as a way to randomize the data stream so that a receive can later recover the clock from the bit transitions and synchronize to the stream. Note that the two bit preamble is never scrambled, otherwise, the synchronization could not be acquired.
\begin{equation}
    P(x) = 1 + x^{39} + x^{58}
    \label{eq:scrambler}
\end{equation}